\chapter{Introduction}
Automated planning, or simply planning, is a branch of Artificial Intelligence. Automated planning is a computational study of the deliberation process, it is the base of information processing tools which provide affordable and efficient planning resources. For over 30 years, many techniques have been developed to solve the planning problems.

Among the techniques, Hierarchical Task Network (HTN) planning is the most widely used one for practical applications. HTN planning has some great advantages: its domain-configurable feature makes HTN planners work efficiently, and allows the planners to solve complex real-world problems; HTN provides a convenient way to write problem-solving methods that correspond to how a human considers. Even though HTN technique has been widely used, we found that the HTN planners lack heuristics to guide their searches.

The goal of this research has been to propose the heuristics used in HTN planning systems, to guide the search to find the best solution of planning problems quickly.

%%%%%%%%%%%%%%%%%%%%%%%%%%%%%%%%%multiagent

The report is organized as follows:

\autoref{chapter1} introduces the principles of automated planning. 

In \autoref{chapter2}, we explain the principles of HTN planning; then an abstract HTN planning process is given, it is also the planning process for which we propose our heuristics; a group of HTN planners we have studied are also introduced in this chapter.

In \autoref{chapter_heuristic}, we propose our heuristics.

In \autoref{chapter4}, my implementation works have been discussed.

Finally, \autoref{chapter5} concludes and discusses the future works.