\begin{thebibliography}{99}
\addcontentsline{toc}{chapter}{Bibliography}
\bibitem{1} M. Ghallab, D. Nau, P. Traverso, Automated Planning: Theory and Practice, Morgan Kaufmann, 2004.  % ’ “” 
\bibitem{2} M. Ghallab, C. Knoblock, D. Wilkins et al., PDDL–the planning domain definition language, the AIPS’98 Planning Competition Committee, 1998.
\bibitem{3} K. Erol, Hierarchical task network planning: Formalization, Analysis and Implementation, Ph.D thesis, Dept. of computer science, Univ. of Maryland, College Park, 1995.
\bibitem{3_1} K. Erol, J. Hendler, and D.S. Nau, UMCP: A Sound and Complete Procedure for Hierarchical Task Network Planning, Proceedings of the Second International Conference, K. J. Hammond, editor, pp. 249–254, Los Altos, CA, Morgan Kaufmann Publishers, Inc, 1994.
\bibitem{3_2} K. Erol, J. Hendler, D.S. Nau, and R. Tsuneto, A Critical Look at Critics in HTN Planning, Proceedings of the 1995 International Joint Conference on Artificial Intelligence (IJCAI-95), pp. 1592-1598, 1995.
\bibitem{4} D. S. Nau, Y. Cao, A. Lotem, and H. Munoz-Avila, SHOP: Simple Hierarchical Ordered Planner, Proceedings of the International Joint Conference on Artificial Intelligence (IJCAI-99), pp.968—973, 1999.
\bibitem{4_1} D.S. Nau, T.-C. Au, O. Ilghami, U. Kuter, D. Wu, Fusun Yaman, H. Munoz-Avila, and W. Murdock. Applications of SHOP and SHOP2. IEEE Intelligent Systems, 20(2):34–41, 2005. Earlier version as Tech. Rep. CS-TR-4604, UMIACS-TR-2004-46.
\bibitem{5} D.S. Nau, Y. Cao, A. Lotem, and H. Munoz-Avila, SHOP and M-SHOP: Planning with Ordered Task Decomposition. Tech. Report CS TR 4157, University of Maryland, College Park, MD, 2000.
\bibitem{godel} V. Shivashankar, R. Alford, U. Kuter, and D. Nau, The GoDeL Planning System: A More Perfect Union of Domain-Independent and Hierarchical Planning. Proceedings of the Twenty-Third international joint conference on Artificial Intelligence (pp. 2380–2386). 2013.
\bibitem{angelic1} B. Marthi, S. Russell, and J. Wolfe, Angelic Semantics for High-Level Actions, in Intl Conf on Automated Planning and Scheduling,
Providence, RI, 2007.
\bibitem{angelic2} B. Marthi, S. J. Russell, and J. Wolfe, Angelic Hierarchical Planning: Optimal and Online Algorithms. ICAPS, 222–231. AAAI Press, 2008.
\bibitem{yoyo} U. Kuter, D.S. Nau, M. Pistore, P. Traverso, Task Decomposition on Abstract States, for Planning under Nondeterminism, Artificial Intelligence 173, 2009.
\bibitem{hipop} P. Bechon, M. Barbier, G. Infantes, C. Lesire, V. Vidal, HiPOP: Hierarchical Partial-Order Planning, 2014.
\bibitem{intro_multi} M. M. de Weerdt and B. J. Clement. Introduction to planning in multiagent systems. Multiagent and Grid Systems An International Journal, 5(4), 2009.
\bibitem{Impact} K. Arisha, F. Ozcan, R. Ross, V. Subrahmanian, T. Eiter, and S. Kraus. IMPACT: A Platform for Collaborating Agents. IEEE Intelligent Systems, 14:64–72, March/April 1999.
\bibitem{ishop} J. Dix, H. Muoz-Avila, D. S. Nau, and L.Zhang, IMPACTing SHOP: Putting an AI Planner into a Multi-Agent Environment. Annals of Mathematics and Artificial Intelligence, 37 (4), 381–407, 2003.
\bibitem{multi1} J.S. Cox, E.H. Durfee, An Efficient Algorithm for Multiagent Plan Coordination, Proceedings of AAMAS, pp. 828–835. ACM, 2005.
\bibitem{multi2} D. Pellier and H. Fiorino. A Unified Framework based on HTN and POP Approaches for Multi-Agent Planning. In International Conference on
Intelligence Agent Technology (IAT), California, USA, 2–5 November
2007.
\bibitem{multi3} D. Pellier, Modèle dialectique pour la synthèse de plans, PhD thesis, UJF - Grenoble, France, 2005. 
\bibitem{astar} P. E. Hart, N. J. Nilsson, B.Raphael, A Formal Basis for the Heuristic Determination of Minimum Cost Paths, IEEE Transactions on Systems \bibitem{PDDL4J} Science and Cybernetics SSC4 4 (2): 100–107, 1968.
D.  Pellier,  PDDL4J,  [Online]  http://sourceforge.net/projects/pdd4j/, 
2011. 



\bibitem{PW92} J. Penberthy and D. Weld. UCPO : A Sound, Complete, Partial Order Planner for ADL. In C. Rich B. Nebel and W. Swartout, editors, Proceedings of the International Conference on Principles of Knowledge Representation and Reasoning, pages 103–114. Morgan Kaufmann Publishers, 1992.
\bibitem{BF97} A. Blum and M. Furst. Fast Planning Through Planning Graph Analysis. Artificial Intelligence, 90(1-2) :281–300, 1997.
\bibitem{RG08} K. Ray and M. Ginsberg. The complexity of optimal planning and a more efficient method for finding solutions. In Proceedings of the International Conference on Automated Planning and Scheduling, pages 280–287, 2008.
\bibitem{KS98a} H. Kautz and B. Selman. The Role of Domain-Specific Knowledge in the Planning as Satisfiability Framework. In Proceedings of the International Conference on Artificial Intelligence Planning and Scheduling, pages 181–189, 1998.
\bibitem{Kam00} S. Kambhampati. Planning graph as a (dynamic) CSP : Exploiting EBL, DDB and other CSP search techniques in graphplan. Journal of Artificial Intelligence Research, 12(1) :1–34, 2000.
\bibitem{PV08} C. Pralet and G. Verfaillie. Using constraint networks on timelines to model and solve planning and scheduling problems. In Proceedings of the International Conference on Automated Planning and Scheduling, pages 272–279, 2008.
\bibitem{MT08} J. Marecki and M. Tambe. Towards faster planning with continuous resources in stochastic domains. In Proceedings of the Association for the Advancement of Artificial Intelligence, pages 1049–1055, 2008.
\bibitem{SWD08} M. Sridharan, J. Wyatt, and R. Dearden. Hippo : Hierarchical pomdps for planning information processing and sensing actions on a robot. In Proceedings of the International Conference on Automated Planning and Scheduling, pages 346–354, 2008.
\bibitem{PT01} M. Pistore and P. Traverso. Planning as Model Checking for Extended Goals in Non-Deterministic Domains. In Proceedings of the International Joint Conference on Artificial Intelligence, pages 479–484, 2001.
\end{thebibliography}